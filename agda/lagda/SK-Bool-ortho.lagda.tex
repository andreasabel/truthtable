%% This file was automatically generated by agda2lagda.

Strong normalization for simply-typed combinatory logic with booleans
using orthogonality.

\heading{Preamble}

\begin{code}
{-# OPTIONS --postfix-projections #-}
{-# OPTIONS --rewriting #-}
{-# OPTIONS --sized-types #-}

open import Agda.Builtin.Equality
open import Agda.Builtin.Size

{-# BUILTIN REWRITE _≡_ #-}

variable i j : Size
\end{code}

Equality lemma

\begin{code}
cong : ∀ {A B : Set} (f : A → B) {x y} → x ≡ y → f x ≡ f y
cong f refl = refl
\end{code}

\heading{Syntax}

Types

\begin{code}
infixr 6 _⇒_

data Ty : Set where
  o     : Ty
  bool  : Ty
  _⇒_   : (a b : Ty) → Ty

variable a b c d : Ty
\end{code}

Types of constants K and S

\begin{code}
K-ty : (a b : Ty) → Ty
K-ty a b = a ⇒ (b ⇒ a)

S-ty : (c a b : Ty) → Ty
S-ty c a b = (c ⇒ (a ⇒ b)) ⇒ (c ⇒ a) ⇒ c ⇒ b
\end{code}

Heads

\begin{code}
data Hd : (d : Ty) → Set where
  K   : Hd (K-ty a b)
  S   : Hd (S-ty c a b)
  tt  : Hd bool
  ff  : Hd bool

variable h : Hd a
\end{code}

Terms and stacks

\begin{code}
infixl 5 _∙_ _∘_
infixr 6 _∷_ _++_

mutual
\end{code}

  Terms are heads under a stack

\begin{code}
  data Tm : (d : Ty) → Set where
    _∙_ : (h : Hd a) (E : Stack i a c) → Tm c

  variable t t′ u u′ v v′ t' u' v' : Tm a
\end{code}

  Single elimination for function or boolean

\begin{code}
  data Elim : (a c : Ty) → Set where
    app   : (u : Tm a)    → Elim (a ⇒ b)  b
    case  : (u v : Tm b)  → Elim bool     b

  variable e e′ e' e₀ e₁ e₂ : Elim a c
\end{code}

  Stacks of eliminations (sized type).

\begin{code}
  data Stack (i : Size) : (a c : Ty) → Set where
    ε    : Stack i c c
    _∷_  : {j : Size< i} (e : Elim a b) (E : Stack j b c) → Stack i a c
\end{code}

%% 
%% data Stack : (i : Size) (a c : Ty) → Set where
%%   ε    : Stack (↑ i) c c
%%   _∷_  : (e : Elim a b) (E : Stack i b c) → Stack (↑ i) a c

\begin{code}
  variable E E' E′ E₀ E₁ E₂ E₃ : Stack i a c
\end{code}

Stack concatenation

\begin{code}
_++_ : Stack ∞ a b → Stack ∞ b c → Stack ∞ a c
ε        ++ E′ = E′
(u ∷ E)  ++ E′ = u ∷ (E ++ E′)

postulate
  ++-ε : (E ++ ε) ≡ E
\end{code}

%% Sized types trouble, see Abel/Vezzosi/Winterhalter, ICFP 2017
%% ++-ε : {E : Stack ∞ a b} → _≡_ {A = Stack ∞ a b} (E ++ ε) E
%% ++-ε {E = ε}     = refl
%% -- ++-ε {E = u ∷ E} rewrite ++-ε {E = E} = refl
%% ++-ε {E = u ∷ E} = {!++-ε {E = E}!}
%%   -- cong (_∷_ {∞} u) (++-ε {E = E})

\begin{code}
++-assoc : (E₁ ++ E₂) ++ E₃ ≡ E₁ ++ (E₂ ++ E₃)
++-assoc {E₁ = ε}       = refl
++-assoc {E₁ = u ∷ E₁}  = cong (u ∷_) (++-assoc {E₁ = E₁})

{-# REWRITE ++-ε ++-assoc #-}
\end{code}

Plugging a term into a stack

\begin{code}
_∘_ : Tm a → Stack ∞ a c → Tm c
h ∙ E ∘ E′ = h ∙ E ++ E′

postulate
  app-ε : t ∘ ε ≡ t
\end{code}

%% 
%% app-ε : t ∘ ε ≡ t
%% app-ε {t = h ∙ E} = {! refl !} -- ++-ε

\begin{code}
app-app : t ∘ E ∘ E′ ≡ t ∘ E ++ E′
app-app {t = h ∙ E} = refl  -- ++-assoc

{-# REWRITE app-ε app-app #-}
\end{code}

\heading{Reduction}

Reduction relations

\begin{code}
infix 4 _↦_ _↦⁺_ _↦ₑ_ _↦ₛ_
\end{code}

One-step reduction in terms and stacks

\begin{code}
mutual

  data _↦_ : (t t′ : Tm a) → Set where
    ↦K   : K   ∙ app t ∷ e              ∷ E ↦ t  ∘ E
    ↦S   : S   ∙ app t ∷ app u ∷ app v  ∷ E ↦ t  ∘ app v ∷ app (u ∘ app v ∷ ε) ∷ E
    ↦tt  : tt  ∙ case u v               ∷ E ↦ u  ∘ E
    ↦ff  : ff  ∙ case u v               ∷ E ↦ v  ∘ E
    ↦E   : (r : E ↦ₛ E′) → h ∙ E ↦ h ∙ E′

  data _↦ₑ_ : (e e' : Elim a c) → Set where
    ↦app    : (r : t ↦ t')  → app {a} {b} t  ↦ₑ app t'
    ↦caseₗ  : (r : u ↦ u')  → case u v       ↦ₑ case u' v
    ↦caseᵣ  : (r : v ↦ v')  → case u v       ↦ₑ case u v'
\end{code}

  Contains single frame permutation

\begin{code}
  data _↦ₛ_ : (E E′ : Stack i a c) → Set where
    π      : {E : Stack i a c} → (case u v ∷ e ∷ E) ↦ₛ (case (u ∘ e ∷ ε) (v ∘ e ∷ ε) ∷ E)
    here   : (r : e ↦ₑ e′)  → e ∷ E ↦ₛ e′ ∷ E
    there  : (r : E ↦ₛ E′)  → e ∷ E ↦ₛ e  ∷ E′
\end{code}

Closure properties of one-step reduction

Concatenation ++ is a congruence

\begin{code}
++↦ₗ : E ↦ₛ E′ → E ++ E₁ ↦ₛ E′ ++ E₁
++↦ₗ π          = π
++↦ₗ (here  r)  = here r
++↦ₗ (there r)  = there (++↦ₗ r)

++↦ᵣ : ∀ (E : Stack i a b) → E₁ ↦ₛ E₂ → E ++ E₁ ↦ₛ E ++ E₂
++↦ᵣ ε       r = r
++↦ᵣ(u ∷ E)  r = there (++↦ᵣ E r)
\end{code}

Application ∘ is a congruence

\begin{code}
∘↦ₗ : t ↦ t′ → t ∘ E ↦ t′ ∘ E
∘↦ₗ ↦K      = ↦K  -- rewrite app-app
∘↦ₗ ↦S      = ↦S
∘↦ₗ ↦tt     = ↦tt
∘↦ₗ ↦ff     = ↦ff
∘↦ₗ (↦E r)  = ↦E (++↦ₗ r)

∘↦ᵣ : ∀ (t : Tm a) → E ↦ₛ E′ → t ∘ E ↦ t ∘ E′
∘↦ᵣ (_∙_ h E) r = ↦E (++↦ᵣ E r)
\end{code}

Transitive closure

%% infixr 100 _⁺

\begin{code}
data _⁺ {A : Set} (R : A → A → Set) (t v : A) : Set where
  sg   : (r : R t v) → (R ⁺) t v
  _∷_  : {u : A} (r : R t u) (rs : (R ⁺) u v) → (R ⁺) t v

_↦⁺_ : (t t′ : Tm a) → Set
_↦⁺_ = _↦_ ⁺
\end{code}

\heading{Strong normalization}

Predicates

\begin{code}
infix 2 _⊂_

_⊂_ : {A : Set} (P Q : A → Set) → Set
P ⊂ Q = ∀{t} (r : P t) → Q t
\end{code}

Accessibility

\begin{code}
data Acc {A : Set} (R : A → A → Set) (t : A) : Set where
  acc : (h : ∀ {t′} (r : R t t′) → Acc R t′) → Acc R t
\end{code}

A relation is wellfounded if its transitive closure is.

\begin{code}
wf⁻ : {A : Set} {R : A → A → Set} → Acc (R ⁺) ⊂ Acc R
wf⁻ (acc h) = acc λ r → wf⁻ (h (sg r))
\end{code}

If a relation is well-founded, so is its transitive closure.

\begin{code}
wf⁺ : {A : Set} {R : A → A → Set} → Acc R ⊂ Acc (R ⁺)
wf⁺ {R = R} hR = acc (loop hR)
  where
  loop : ∀{t} → Acc R t → (R ⁺) t ⊂ Acc (R ⁺)
  loop (acc h) (sg r)    = wf⁺ (h r)
  loop (acc h) (r ∷ rs)  = loop (h r) rs
\end{code}

Reducts of SN things are SN things

\begin{code}
sn-red : {A : Set} {R : A → A → Set}{t t′ : A} → Acc R t → R t t′ → Acc R t′
sn-red (acc sn) r = sn r
\end{code}

Strong normalization: t is SN if all of its reducts are, inductively.

\begin{code}
SN SN⁺ : Tm a → Set
SN   = Acc _↦_
SN⁺  = Acc _↦⁺_

SNₑ : Elim a c → Set
SNₑ = Acc _↦ₑ_

SNₛ : Stack i a c → Set
SNₛ = Acc _↦ₛ_
\end{code}

Deconstruction of SN t

\begin{code}
sn-spine : SN (h ∙ E) → SNₛ E
sn-spine (acc sntE) = acc λ r → sn-spine (sntE (↦E r))
\end{code}

Constants are SN

Heads are SN

\begin{code}
sn-Hd : SN (h ∙ ε)
sn-Hd = acc λ{ (↦E ()) }
\end{code}

The empty stack is SN

\begin{code}
sn-ε : SNₛ (ε {c = a})
sn-ε = acc λ()
\end{code}

Function elimination preserves SN

\begin{code}
sn-app∷ : SN u → SNₛ E → SNₛ (app u ∷ E)
sn-app∷ (acc snu) snE@(acc snE') = acc λ where
   (here (↦app r))  → sn-app∷ (snu r)    snE
   (there r)        → sn-app∷ (acc snu)  (snE' r)
\end{code}

Underapplied functions are SN

Kt is SN

\begin{code}
sn-Kt : SN t → SN (K {a} {b} ∙ app t ∷ ε)
sn-Kt (acc snt) = acc λ{ (↦E (here (↦app r))) → sn-Kt (snt r) }
\end{code}

St is SN

\begin{code}
sn-St : SN t → SN (S {a} {b} ∙ app t ∷ ε)
sn-St (acc snt) = acc λ{ (↦E (here (↦app r))) → sn-St (snt r) }
\end{code}

Stu is SN

\begin{code}
sn-Stu : SN t → SN u → SN (S {a} {b} ∙ app t ∷ app u ∷ ε)
sn-Stu (acc snt) (acc snu) = acc λ where
   (↦E (here (↦app r)))          → sn-Stu (snt r)    (acc snu)
   (↦E (there (here (↦app r))))  → sn-Stu (acc snt)  (snu r)
\end{code}

Redexes are SN

KtuE is SN

\begin{code}
sn-KtuE : SN (t ∘ E) → SN u → SN (K ∙ app t ∷ app u ∷ E)
sn-KtuE {t = t} sntE@(acc h) (acc snu)  = acc λ where
   ↦K                            → sntE
   (↦E (here (↦app r)))          → sn-KtuE (h (∘↦ₗ r))    (acc snu)
   (↦E (there (here (↦app r))))  → sn-KtuE sntE           (snu r)
   (↦E (there (there r)))        → sn-KtuE (h (∘↦ᵣ t r))  (acc snu)
\end{code}

StuvE is SN

\begin{code}
sn-StuvE : SN⁺ (t ∘ app v ∷ app (u ∘ app v ∷ ε) ∷ E)
          → SN (S ∙ app t ∷ app u ∷ app v ∷ E)
sn-StuvE {t = t} {u = u} sntvuvE@(acc h) = acc λ where
  ↦S →
    wf⁻ sntvuvE

  (↦E (here (↦app r))) →
    sn-StuvE (h (sg (∘↦ₗ r)))

  (↦E (there (here (↦app r)))) →
    sn-StuvE (h (sg (∘↦ᵣ t (there (here (↦app (∘↦ₗ r)))))))

  (↦E (there (there (here (↦app r))))) →
    sn-StuvE (h (∘↦ᵣ t (here (↦app r)) ∷
                 sg (∘↦ᵣ t (there (here (↦app (∘↦ᵣ u (here (↦app r)))))))))

  (↦E (there (there (there r)))) →
    sn-StuvE (h (sg (∘↦ᵣ t (there (there r)))))
\end{code}

This is the key lemma:

\begin{code}
mutual

  sn-case : {E : Stack i a c} (sntE : SN (t ∘ E)) (snuE : SN (u ∘ E)) → SN (h ∙ case t u ∷ E)
  sn-case sntE snuE = acc (sn-case' sntE snuE)
\end{code}

  Case distinction on reductions of (h ∙ case t u ∷ E)

\begin{code}
  sn-case' : {E : Stack i a c}
            (sntE : SN (t ∘ E))
            (snuE : SN (u ∘ E))
            (r : h ∙ case t u ∷ E ↦ v) → SN v
  sn-case'  sntE snuE ↦tt = sntE
  sn-case'  sntE snuE ↦ff = snuE
  sn-case'  (acc sntE) snuE (↦E (here (↦caseₗ r)))   = sn-case (sntE (∘↦ₗ r)) snuE
  sn-case'  sntE (acc snuE) (↦E (here (↦caseᵣ r)))   = sn-case sntE (snuE (∘↦ₗ r))
  sn-case'  {t = t} {u = u}
            (acc sntE) (acc snuE) (↦E (there r))     = sn-case (sntE (∘↦ᵣ t r)) (snuE (∘↦ᵣ u r))
  sn-case'  {i = .(↑ i)} sntE snuE (↦E (π {i = i}))  = sn-case {i = i} sntE snuE
\end{code}

%% -- Internal error with this version (#4929)
%% sn-case : {E : Stack i a c}
%%           (sntE : SN (t ∘ E))
%%           (snuE : SN (u ∘ E))
%%           → SN (h ∙ case t u ∷ E)
%% sn-case {i = i} {t = t} {u = u} (acc sntE) (acc snuE) = acc
%%   λ { ↦tt → acc sntE
%%     ; ↦ff → acc snuE
%%     ; (↦E (here (↦caseₗ r)))  → sn-case (sntE (∘↦ₗ r)) (acc snuE)
%%     ; (↦E (here (↦caseᵣ r))) → sn-case (acc sntE) (snuE (∘↦ₗ r))
%%     ; (↦E (there r))         → sn-case (sntE (∘↦ᵣ t r)) (snuE (∘↦ᵣ u r))
%%     ; (↦E (π {i = j}))       → sn-case {i = j} (acc sntE) (acc snuE)
%%     }

\heading{Semantic types}

Stack sets

\begin{code}
record Cont a : Set where
  constructor cont
  field
    {len}  : Size
    {tgt}  : Ty
    st     : Stack len a tgt

Predₛ : (a : Ty) → Set₁
Predₛ a = (cE : Cont a) → Set

variable A B C D : Predₛ a
\end{code}

Elementhood in stack sets

\begin{code}
infix 2 _∈_

_∈_ : (E : Stack i a c) (A : Predₛ a) → Set
E ∈ A = A (cont E)
\end{code}

Semantic objects

We use a record to help Agda's unifier.

\begin{code}
record _⊥_ (t : Tm a) (A : Predₛ a) : Set where
  field run : (⦅E⦆ : E ∈ A) → SN (t ∘ E)
open _⊥_

module Remark1 where
\end{code}

  Semantic objects are closed under reduction.

\begin{code}
  sem-red : t ⊥ A → t ↦ t′ → t′ ⊥ A
  sem-red ⦅t⦆ r .run ⦅E⦆ = sn-red (⦅t⦆ .run ⦅E⦆) (∘↦ₗ r)
\end{code}

Singleton stack set {ε}

\begin{code}
data ⟦o⟧ {a} : Predₛ a where
  ε : ε ∈ ⟦o⟧
\end{code}

Semantic booleans

\begin{code}
record ⟦bool⟧ (cE : Cont bool) : Set where
  field br : let cont E = cE in ∀ h → SN (h ∙ E)
open ⟦bool⟧
\end{code}

Boolean values

\begin{code}
⦅tt⦆ : (tt ∙ ε) ⊥ ⟦bool⟧
⦅tt⦆ .run ⦅E⦆ = ⦅E⦆ .br tt

⦅ff⦆ : (ff ∙ ε) ⊥ ⟦bool⟧
⦅ff⦆ .run ⦅E⦆ = ⦅E⦆ .br ff
\end{code}

Interpretation of case

\begin{code}
⦅case⦆ : (⦅t⦆ : t ⊥ A) (⦅u⦆ : u ⊥ A) (⦅E⦆ : E ∈ A) → case t u ∷ E ∈ ⟦bool⟧
⦅case⦆ ⦅t⦆ ⦅u⦆ ⦅E⦆ .br h = sn-case (⦅t⦆ .run ⦅E⦆) (⦅u⦆ .run ⦅E⦆)
\end{code}

Function space on semantic types

\begin{code}
data _⟦→⟧_ (A : Predₛ a) (B : Predₛ b) : Predₛ (a ⇒ b) where
  ε    : ε ∈ (A ⟦→⟧ B)
  _∷_  : (⦅u⦆ : u ⊥ A) (⦅E⦆ : E ∈ B) → (app u ∷ E) ∈ (A ⟦→⟧ B)
\end{code}

Application

\begin{code}
⦅app⦆ : (⦅t⦆ : t ⊥ (A ⟦→⟧ B)) (⦅u⦆ : u ⊥ A) → (t ∘ app u ∷ ε) ⊥ B
⦅app⦆ ⦅t⦆ ⦅u⦆ .run ⦅E⦆ = ⦅t⦆ .run (⦅u⦆ ∷ ⦅E⦆)
\end{code}

\heading{Semantic Types}

Semantic types are specified by sets of SN stacks that contain ε.

\begin{code}
record SemTy (A : Predₛ a) : Set where
  field
    id  : ε ∈ A
    sn  : (⦅E⦆ : E ∈ A) → SNₛ E
open SemTy

Sem-sn : (⟨A⟩ : SemTy A) (⦅t⦆ : t ⊥ A) → SN t
Sem-sn ⟨A⟩ ⦅t⦆ = ⦅t⦆ .run (⟨A⟩ .id)
\end{code}

%% 
%% Sem-snₑ : (⟨A⟩ : SemTy A) (⦅E⦆ : E ∈ A) → SNₛ E
%% Sem-snₑ ⟨A⟩ = ⟨A⟩ .sn

SN is the semantic type given by {ε}

\begin{code}
⟨o⟩ : SemTy (⟦o⟧ {a = a})
⟨o⟩ .id    = ε
⟨o⟩ .sn ε  = sn-ε

⟨bool⟩ : SemTy ⟦bool⟧
⟨bool⟩ .id .br h  = sn-Hd
⟨bool⟩ .sn  ⦅E⦆    = sn-spine (⦅E⦆ .br tt)

_⟨→⟩_ : (⟨A⟩ : SemTy A) (⟨B⟩ : SemTy B) → SemTy (A ⟦→⟧ B)
(⟨A⟩ ⟨→⟩ ⟨B⟩) .id            = ε
(⟨A⟩ ⟨→⟩ ⟨B⟩) .sn ε          = sn-ε
(⟨A⟩ ⟨→⟩ ⟨B⟩) .sn (⦅u⦆ ∷ ⦅E⦆)  = sn-app∷ (Sem-sn ⟨A⟩ ⦅u⦆) (⟨B⟩ .sn ⦅E⦆)
\end{code}

\heading{Soundness}

Type interpretation

\begin{code}
⟦_⟧ : ∀ a → Predₛ a
⟦ o ⟧      = ⟦o⟧
⟦ bool ⟧   = ⟦bool⟧
⟦ a ⇒ b ⟧  = ⟦ a ⟧ ⟦→⟧ ⟦ b ⟧
\end{code}

Types are interpreted as semantic types

\begin{code}
⟨_⟩ : ∀ a → SemTy ⟦ a ⟧
⟨ o     ⟩  = ⟨o⟩
⟨ bool  ⟩  = ⟨bool⟩
⟨ a ⇒ b ⟩  = ⟨ a ⟩ ⟨→⟩ ⟨ b ⟩
\end{code}

Semantic objects are SN

\begin{code}
sem-sn : t ⊥ ⟦ a ⟧ → SN t
sem-sn {a = a} ⦅t⦆ = Sem-sn ⟨ a ⟩ ⦅t⦆
\end{code}

%% 
%% sem-snₛ : E ∈ ⟦ a ⟧ → SNₛ E
%% sem-snₛ {a = a} ⦅E⦆ = ⟨ a ⟩ .sn ⦅E⦆

Interpretation of K

\begin{code}
⦅K⦆ : (K ∙ ε) ⊥ ⟦ K-ty a b ⟧
⦅K⦆ .run ε                  = sn-Hd
⦅K⦆ .run (⦅t⦆ ∷ ε)          = sn-Kt (sem-sn ⦅t⦆)
⦅K⦆ .run (⦅t⦆ ∷ ⦅u⦆ ∷ ⦅E⦆)  = sn-KtuE (⦅t⦆ .run ⦅E⦆) (sem-sn ⦅u⦆)
\end{code}

%% ⦅K⦆ .run (⦅t⦆ ∷ ⦅u⦆ ∷ ⦅E⦆)  = sn-KtuE (⦅t⦆ .run ⦅E⦆) (sem-sn ⦅t⦆) (sn-app (sem-sn ⦅u⦆)) (sem-snₛ ⦅E⦆)

Interpretation of S

\begin{code}
⦅S⦆ : (S ∙ ε) ⊥ ⟦ S-ty c a b ⟧
⦅S⦆ .run ε                        = sn-Hd
⦅S⦆ .run (⦅t⦆ ∷ ε)                = sn-St (sem-sn ⦅t⦆)
⦅S⦆ .run (⦅t⦆ ∷ ⦅u⦆ ∷ ε)          = sn-Stu (sem-sn ⦅t⦆) (sem-sn ⦅u⦆)
⦅S⦆ .run (⦅t⦆ ∷ ⦅u⦆ ∷ ⦅v⦆ ∷ ⦅E⦆)  = sn-StuvE (wf⁺ (⦅t⦆ .run (⦅v⦆ ∷ ⦅app⦆ ⦅u⦆ ⦅v⦆ ∷ ⦅E⦆)))
\end{code}

%% 
%% ⦅S⦆ .run (⦅t⦆ ∷ ⦅u⦆ ∷ ⦅v⦆ ∷ ⦅E⦆)  =
%%   sn-StuvE
%%    (⦅t⦆ .run (⦅v⦆ ∷ ⦅app⦆ ⦅u⦆ ⦅v⦆ ∷ ⦅E⦆))
%%    (sem-sn ⦅t⦆) (sem-sn ⦅u⦆) (sem-sn ⦅v⦆) (sem-snₛ ⦅E⦆)

Interpretation of constants

\begin{code}
⦅_⦆ₕ : (h : Hd a) → (h ∙ ε) ⊥ ⟦ a ⟧
⦅ K ⦆ₕ   = ⦅K⦆
⦅ S ⦆ₕ   = ⦅S⦆
⦅ tt ⦆ₕ  = ⦅tt⦆
⦅ ff ⦆ₕ  = ⦅ff⦆
\end{code}

Term interpretation (soundness)

\begin{code}
mutual
  ⦅_⦆ : (t : Tm a) → t ⊥ ⟦ a ⟧
  ⦅ h ∙ E ⦆ .run ⦅E′⦆ = ⦅ h ⦆ₕ .run (⦅ E ⦆ₛ ⦅E′⦆)

  ⦅_⦆ₛ : (E : Stack i a c) (⦅E'⦆ : E' ∈ ⟦ c ⟧) → (E ++ E') ∈ ⟦ a ⟧
  ⦅ ε {c = a}    ⦆ₛ  ⦅E'⦆ = ⦅E'⦆
  ⦅ app u ∷ E    ⦆ₛ  ⦅E'⦆ = ⦅ u ⦆ ∷ ⦅ E ⦆ₛ ⦅E'⦆
  ⦅ case u v ∷ E ⦆ₛ  ⦅E'⦆ = ⦅case⦆ ⦅ u ⦆ ⦅ v ⦆ (⦅ E ⦆ₛ ⦅E'⦆)
\end{code}

Strong normalization

\begin{code}
thm : (t : Tm a) → SN t
thm t = sem-sn ⦅ t ⦆
\end{code}


