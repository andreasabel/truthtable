\section{Saturated Sets}
\label{sec:sat}

In this appendix, we show how to adapt the original \emph{saturated
  sets} method to IITND, first just for $\beta$-SN, then including
$\pi$-reductions.

\subsection{Saturated Sets for Computation Reductions}
\label{sec:satbeta}

In the following, we adapt Tait's method of saturated sets to show $\beta$-SN
for ITTND.  This is a variation of the proof by Geuvers and Hurkens
\cite{geuversHurkens:types17}.

We first observe that the set $\SN$ contains a weak-head
redex already when (1) all of its reducts are SN and (2) its \emph{lost terms}
are SN, where a lost term is a subterm that could get dropped by all of the
weak-head reductions.
This fact is made precise by the following lemma:
\begin{lemma}
  \label{lem:clos-sn}
  The following implications, written as rules, are valid closure
  properties of $\SN$:
\begin{gather*}
  \ru{t_1[a] \cdot \el\to{10}(a,u) \cdot \vec E \in \SN
      \qquad
      t_2 \in \SN
    }{\inn\to{00}(t_1,t_2) \cdot \el\to{10}(a,u) \cdot \vec E \in \SN}
\\[2ex]
  \ru{t[a] \cdot \el\to{10}(a,u) \cdot \vec E \in \SN
      \qquad
      u[b] \cdot \vec E \in \SN
      \qquad
      b \in \SN
    }{\inn\to{01}(t,b) \cdot \el\to{10}(a,u) \cdot \vec E \in \SN}
\qquad
  \ru{u[b] \cdot \vec E \in \SN
      \qquad
      a_1,a_2,b \in \SN
    }{\inn\to{11}(a_1,b) \cdot \el\to{10}(a_2,u) \cdot \vec E \in \SN}
\end{gather*}
(Spine $\vec E$ may be empty in all cases.)
\end{lemma}
\begin{proof}
  Each of these implications is proven by induction on the premises,
  establishing that the possible reducts of the term in the conclusion
  are SN.
  The weak-head reduct(s) are covered by the premises in each case.
  % Reductions in $\vec E$ are
  % covered by a side induction on $\vec E \in \SN$.
  Reductions in lost terms are covered by the extra SN hypotheses.
  Reductions in preserved terms are covered by the main SN hypotheses.
  (This includes reductions in the spine $\vec E$.)


  For example, consider the case for $\inn\to{00}$:
  By induction on $t_1[a] \cdot \el\to{10}(a,u) \cdot \vec E \in \SN$
  and $t_2 \in \SN$ show $t' \in \SN$ given
  $\inn\to{00}(t_1,t_2) \cdot \el\to{10}(a,u) \cdot \vec E \red t'$.
  \begin{caselist}

  \nextcase
  $t' = t_1[a] \cdot \el\to{10}(a,u) \cdot \vec E$.
  Then $t' \in \SN$ by assumption.

  \nextcase
  $t' = \inn\to{00}(t_1,t_2') \cdot \el\to{10}(a,u) \cdot \vec E$
  where $t_2 \red t_2'$.
  Then $t' \in \SN$ by induction hypothesis $t_2' \in \SN$.

  \nextcase
  $t' = \inn\to{00}(t_1',t_2) \cdot \el\to{10}(a',u') \cdot \vec E'$
  where $(t_1,a,u,\vec E) \red (t_1',a',u',\vec E')$ (a single
  reduction in one of these subterms).
  Then $t_1[a] \cdot \el\to{10}(a,u) \cdot \vec E \red^+
  t_1'[a'] \cdot \el\to{10}(a',u') \cdot \vec E'$ (several steps
  possible, \eg, if reduction was in $a$ and $t_1$ mentions the 0th
  de~Bruijn index).
  Thus, $t' \in \SN$ by induction hypothesis on
  $t_1'[a'] \cdot \el\to{10}(a',u') \cdot \vec E' \in \SN$.
  \popQED
  \end{caselist}
\end{proof}

% A term set shall be called \emph{saturated} if it is closed under weak
% head expansion.

Mimicking \Cref{lem:clos-sn},
the \emph{saturation} $\cl\A$ of a term set is---in the case of the
implicational fragment of ITTND--- defined inductively as follows:
\begin{gather*}
  \ru{t \in \A_\Gamma
    }{t \in \cl\A_\Gamma}
\qquad
  \ru{t_1[a] \cdot \el\to{10}(a,u) \cdot \vec E \in \cl\A_\Gamma
      \qquad
      t_2 \in \SN
    }{\inn\to{00}(t_1,t_2) \cdot \el\to{10}(a,u) \cdot \vec E \in \cl\A_\Gamma}
\\[2ex]
  \ru{t[a] \cdot \el\to{10}(a,u) \cdot \vec E \in \cl\A_\Gamma
      \qquad
      u[b] \cdot \vec E \in \cl\A_\Gamma
      \qquad
      b \in \SN
    }{\inn\to{01}(t,b) \cdot \el\to{10}(a,u) \cdot \vec E \in \cl\A_\Gamma}
\qquad
  \ru{u[b] \cdot \vec E \in \cl\A_\Gamma
      \qquad
      a_1,a_2,b \in \SN
    }{\inn\to{11}(a_1,b) \cdot \el\to{10}(a_2,u) \cdot \vec E \in \cl\A_\Gamma}
\end{gather*}
%(Spine $\vec E$ may be empty in all cases.)


\begin{lemma}
  \label{lem:sat-preserves-sn}
  $\cl\SN \subseteq \SN$.
\end{lemma}
\begin{proof}
  We show $t \in \SN$ by induction on $t \in \cl\SN$, using \Cref{lem:clos-sn}.
\end{proof}
% The design principle is that the closure operation adds a weak-head
% redex to a term set when (1) all of its reducts are already in the set or
% have been added in the closure process, and (2) its \emph{lost terms}
% are SN.  A lost term is a subterm that gets dropped by a reduction.
% Without requirement (2), the closure operation would not preserve SN, but
% now it does:

% \begin{lemma}
%   \label{lem:sat-preserves-sn}
%   $\cl\SN \subseteq \SN$.
% \end{lemma}
% \begin{proof}
%   We show by induction on $t \in \cl\SN$ that all reducts of $t$ are
%   SN.  By design of the closure operation, the weak-head reducts are
%   covered by the induction hypothesis.
%   % Reductions in $\vec E$ are
%   % covered by a side induction on $\vec E \in \SN$.
%   Reductions in lost terms are covered by the extra SN hypotheses.
%   Reductions in preserved terms are covered by induction on SN
%   obtained from the main hypotheses.  This includes reductions in the
%   spine $\vec E$.

%   For example, consider the case for $\inn\to{00}$:
%   \[
%     \ru{t_1[a] \cdot \el\to{10}(a,u) \cdot \vec E \in \cl\SN
%         \qquad
%         t_2 \in \SN
%       }{\inn\to{00}(t_1,t_2) \cdot \el\to{10}(a,u) \cdot \vec E \in \cl\SN}
%   \]
%   We locally induct on $t_2 \in \SN$ and
%   $t_1[a] \cdot \el\to{10}(a,u) \cdot \vec E \in \cl\SN$
%   (where the latter is the main induction hypothesis)
%   to show that each reduct of
%   $\inn\to{00}(t_1,t_2) \cdot \el\to{10}(a,u) \cdot \vec E$ is SN.
%   \begin{caselist}

%   \nextcase
%   $\inn\to{00}(t_1,t_2) \cdot \el\to{10}(a,u) \cdot \vec E \red
%    t_1[a] \cdot \el\to{10}(a,u) \cdot \vec E \in \SN$ by main
%    induction hypothesis.
%   \nextcase
%   $\inn\to{00}(t_1,t_2) \cdot \el\to{10}(a,u) \cdot \vec E \red
%    \inn\to{00}(t_1,t_2') \cdot \el\to{10}(a,u) \cdot \vec E$
%   where $t_2 \red t_2'$
%   \end{caselist}

% \end{proof}

\begin{corollary}
  \label{cor:sat-preserves-sn}
  If $\A \subseteq \SN$ then $\cl\A \subseteq \SN$.
\end{corollary}
\begin{proof}
  Since closure is a monotone operator, we have $\cl\A \subseteq
  \cl\SN \subseteq \SN$ by \Cref{lem:sat-preserves-sn}.
\end{proof}

A \emph{saturated set} $\A \in \SAT$ must fulfill the following three
properties:
\begin{itemize}%[leftmargin=3em]
\setlength{\itemindent}{2.7em}

\item[\namedlabel{it:sat1}{SAT1}]
$\A \subseteq \SN$ (contains only SN terms).

\item[\namedlabel{it:sat2}{SAT2}]
If $\vec E \in \SN$ then $x \cdot \vec E \in \A$ (contains SN
neutrals).

\item[\namedlabel{it:sat3}{SAT3}]
$\cl\A \subseteq \A$ (closed under SN weak-head expansion).

\end{itemize}

Semantic implication can now be defined as:
\[
  f \in (\A \to \B)_\Gamma
  \iff
  f \in \SN \mbox{ and }
  \forall \C \in \SAT, \,
  \tau : \Delta \leq \Gamma, \,
  a \in \A_\Delta, \,
  t \in \C[\B]_\Delta.\
  f\tau \cdot \el\to{10}(a,t) \in \C_\Delta .
\]

\begin{lemma}[Function space on $\SAT$]
  \label{lem:fun-sat}
  If $\A \subseteq \SN$ and $\B \in \SAT$, then $\A \to \B \in \SAT$.
\end{lemma}
\begin{proof}
  \ref{it:sat1} holds by definition.
  \ref{it:sat2} holds by \ref{it:sat2} of $\B$.
  \ref{it:sat3} holds by \ref{it:sat3} of $\B$.
\end{proof}

The introductions rules for implication are indeed modeled for the
$\SAT$ variant of semantic function space.  For instance, $\inn\to{01}$:
\begin{lemma}[Introduction \ref{it:in01}]
  \label{lem:intro-sat}
  If $t \in (\A \to \B)[\A]\indy\Gamma$
  and $b \in \B\indy\Gamma$ then
  $\inn\to{01}(t,b) \in \indyp{\A \to \B}{\Gamma}$.
\end{lemma}
\begin{proof}
  Assume
  $\C \in \SAT$ and
  $\tau : \Delta \leq \Gamma$ and
  $a \in \A\indy\Delta$ and $u \in \C[\B]\indy\Delta$ and
  show
  $\inn\to{01}(t,b)\tau \cdot \el\to{10}(a,u) \in \C\indy\Delta$.
  Using \ref{it:sat3} on $\C$, it is sufficient to show that
  (1) $t(\tau.a) \cdot \el\to{10}(a,u) \in \C\indy\Delta$ and
  (2) $u[b\tau] \in \C\indy\Delta$ and
  (3) $b\tau \in \SN$.
  Subgoals (2) and (3) follow since $b\tau \in \B\indy\Delta$,
  and (1) holds since $t(\tau.a) \in (\A \to \B)\indy\Delta$.
\end{proof}


\subsection{On Permutation Reductions}
\label{sec:sat-perm}

Ralph Matthes' \cite{matthes:classlog} formulation of saturated sets in
the context of $\pi$-reductions can also be adapted to ITTND.

First, we observe that \Cref{lem:clos-sn} still holds if
$\pi$-reductions are taken into account.  This is because any
reduction in the spine of a conclusion can be simulated in the spine
of at least one of the premises.

Thus, \ref{it:sat3} can remain in place, only \ref{it:sat2} needs to
be reformulated, since a neutral $x \cdot \vec E$ can be subject to a
$\beta$-reduction after a $\pi$-reduction in $\vec E$ has created a
new $\beta$-redex.  Towards a reformulation of \ref{it:sat2}, we
observe the following closure properties of SN by neutral terms:
\begin{lemma}[Neutral closure of $\SN$]
  \label{lem:clos-ne}
  The following implications, written as rules, are valid closure
  properties of $\SN$:
\begin{gather*}
  \ru{}{x \in \SN}
\qquad
  \ru{a \in \SN \qquad u \in \SN
    }{x \cdot \el\to{10}(a,u) \in \SN}
\qquad
  \ru{x \cdot E_1\{E_2\} \cdot \vec E \in \SN \qquad
      E_2 \cdot \vec E \in \SN
    }{x \cdot E_1 \cdot E_2 \cdot \vec E \in \SN}
\end{gather*}
\end{lemma}
The extra assumption $E_2 \cdot \vec E \in \SN$ in the third
implication is equivalent to $y \cdot E_2 \cdot \vec E \in \SN$ for
some variable $y$.
In the implicational fragment, this assumption is
redundant since the composition $\el\to{10}(a,u)\{E_2\} =
\el\to{10}(a,u \cdot E_2\up)$ does not lose $E_2$.
In particular, any reduction in $E_2 \cdot \vec E$ can be replayed in
$x \cdot E_1\{E_2\} \cdot \vec E$.
However, in general there can be eliminations with only
positive premises, such as $\el\neg{1}(a)$ for negation, where composition
$\el\neg{1}(a)\{E_2\} = \el\neg{1}(a)$ simply drops $E_2$.
This means that reductions in part $E_2 \cdot \vec E$ of
$x \cdot E_1 \cdot E_2 \cdot \vec E$ cannot necessarily be simulated
in $x \cdot E_1\{E_2\} \cdot \vec E$.  In particular, a reduction $E_2
\cdot E_3 \red[\pi] E_2\{E_3\}$ could lead to new $\beta$-redexes which
have no correspondence in $E_1\{E_2\} \cdot E_3$.

\newcommand{\rvar}{\rulename{var}}
\newcommand{\rel}{\rulename{el}}
\newcommand{\rpi}{\rulename{pi}}

Mimicking \Cref{lem:clos-ne},
we \emph{extend} the definition of saturation $\cl\C$ of a semantic type $\C$
for $C$ by the following three clauses:
\begin{gather*}
  \nru{\rvar}{x : \Gamma \vdash C}{x \in \cl\C_\Gamma}
\qquad
  \nru{\rel
    }{x : \Gamma \vdash A \to B \qquad
      a \in \SN(\Gamma \vdash A) \qquad
      u \in \cl\C_{\Gamma.B}
    }{x \cdot \el\to{10}(a,u) \in \cl\C_\Gamma}
\\[2ex]
  \nrul{\rpi
    }{x : \Gamma \vdash A \qquad
      E_1 : \Gamma \mid A \vdash B \qquad
      x \cdot E_1\{E_2\} \cdot \vec E \in \cl\C_\Gamma \\
      \tau : \Delta \leq \Gamma \qquad
      y : \Delta \vdash B \qquad
      y \cdot (E_2 \cdot \vec E)\tau \in \cl\C_{\Delta}
      % \x_0 \cdot E_2\up \cdot \vec E\up \in \cl\C_{\Gamma.B}
    }{x \cdot E_1 \cdot E_2 \cdot \vec E \in \cl\C_\Gamma}
\end{gather*}
Note that a premise such as $y \cdot (E_2 \cdot \vec E)\tau \in \SN$ would
be too weak to show that semantic function space is saturated.

We revise the definition of $\SAT$ such that \ref{it:sat3} uses the
extended definition of closure, obsoleting \ref{it:sat2}.


\begin{lemma}[Function space on $\SAT$]
  \label{lem:fun-sat-pi}
  If $\A \subseteq \SN$ and $\B \in \SAT$, then $\A \to \B \in \SAT$.
\end{lemma}
\begin{proof}
  We shall focus on the new closure conditions for $\SAT$:
\begin{itemize}
\item \rvar: Show $x \in (\A \to \B)_\Gamma$.  Clearly $x \in \SN$.
  %
  Now assume $\C \in \SAT$ and $\tau : \Delta \leq \Gamma$ and
  $a \in \A_\Delta$ and $u \in \C[\B]_\Delta$ and show
  $x\tau \cdot \el\to{10}(a,u) \in \C_\Delta$.
  %
  By \rel, it is sufficient that $a \in \SN$ and
  $u \in \C_{\Delta.B}$.
  By \rvar we have $\x_0 \in \B_{\Delta.B}$,
  thus $u(\up.\x_0) = u \in \C_{\Delta.B}$.

\item \rel: Assume $x : \Gamma \vdash A_0 \to B_0$ and
  $a_0 \in \SN(\Gamma \vdash A_0)$ and
  $u_0 \in \cl{\A \to \B}_{\Gamma.B_0}$ and show
  $x \cdot \el\to{10}(a_0,u_0) \in \cl{\A \to \B}_\Gamma$.
  First, $x \cdot \el\to{10}(a_0,u_0) \in \SN$.

  Further, assume $\C \in \SAT$ and $\tau : \Delta \leq \Gamma$
  and $a \in \A_\Delta$ and $u \in \C[\B]_\Delta$ and show
  $(x \cdot \el\to{10}(a_0,u_0))\tau \cdot \el\to{10}(a,u) \in \C_\Delta$.
  Using \rpi, we first discharge the last subgoal $\x_0 \cdot
  \el\to{10}(a,u)\up \in \C_{\Delta.(A \to B)}$ by \rel for $\C$ with
  $a\up \in \A_{\Delta.(A \to B)}$ and $u(\Up\up) \in \C_{\Delta.(A\to
    B).B}$.

  It remains to show that $x\tau \cdot \el\to{10}(a_0\tau, u_0(\Up\tau) \cdot
  \el\to{10}(a,u)\up) \in \C_\Delta$.  Again, we use \rel for $\C$.
  Clearly $a_0\tau \in \SN$, so it remains to show that
  $u_0(\Up\tau) \cdot \el\to{10}(a\up,u(\Up\up)) \in \C_{\Delta.B_0}$.
  Since $u_0(\Up\tau) \in (\A \to \B)_{\Delta.B_0}$
  and $a\up \in \A_{\Delta.B_0}$ and $u(\Up\up) \in
  \C[\B]_{\Delta.B_0}$,
  this is the case by definition of $\A \to \B$.

\item \rpi:
  The case \rpi for $\A \to \B$ is shown by $\rpi$ for $\C$ (what $\C$
  refers to, see the previous cases).  This part is a bit tedious to
  spell out, but completely uninteresting, since just $\vec E$ is extended
  by another $\el\to{10}$-elimination at the end.
%
\popQED
\end{itemize}
\end{proof}

The soundness of the introductions carries over from the previous
section (\Cref{lem:intro-sat}) since the saturated sets are still
closed by weak head expansion.

This concludes the $\beta\pi$-SN proof for ITTND using saturated sets.

%%% Local Variables:
%%% mode: latex
%%% TeX-master: "types-post2020"
%%% End:
